\documentclass[10pt,letterpaper,oneside]{article}
\usepackage[utf8]{inputenc}
\usepackage{amsmath}
\usepackage{amsfonts}
\usepackage{amssymb}
\usepackage{graphicx}

\author{Alex M. Chubaty}
\title{Population genetics equations}

\begin{document}
\section{Mutation}
\paragraph()
Here we describe a biallelic system where $A$ mutates to $a$ at some frequency $\mu$, and $a$ mutates to $A$ at frequency $\nu$

\begin{equation}
	p' = (1-\mu)p + (1-p)\nu
\end{equation}


\section{Migration}
\paragraph()
Here we describe a biallelic system where individuals move from a continent (mainland), with an allele frequency of $p_C$, to an island at a net migration rate of $m$. The allele frequency on the island $p_I$.

\begin{equation}
	p'_I = (1-m)p_I + m p_C
\end{equation}


\section{Selection}
\paragraph()
Here we describe a biallelic system where selection acts on each genotype based on the value of the relative fitness of each genotype ($w_{AA}$, $w_{Aa}$, $w_{aa}$).

\begin{equation}
	p' = \frac{w_{AA} p^2 + w_{Aa} p (1-p)}{w_{AA} p^2 + w_{Aa} 2 p (1-p) + w_{aa} (1-p)^2}
\end{equation}

\end{document}